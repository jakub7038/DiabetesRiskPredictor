\chapter{Przegląd Aktualnych Rozwiązań i Analiza Wymagań}
\label{cha:analiza}

\section{Przegląd aktualnych rozwiązań}

\subsection{Formularz American Diabetes Association}

Internetowy kalkulatory ryzyka cukrzycy - formularz American Diabetes Association jest jednym z najczęściej spotykanych rozwiązań pozwalających na szybką samoocenę stanu zdrowia. Umożliwiają one wprowadzenie podstawowych biometrycznych, takich jak wiek, waga czy wzrost, w celu oszacowania ryzyka zachorowania w oparciu o sztywne reguły punktowe.

\begin{figure}[htbp]
    \centering
    \includegraphics[width=0.8\textwidth]{figures/formularzamericanda1.jpg}
    \caption{Formularz American Diabetes Association}
    \label{fig:ada_form}
\end{figure}

Do głównych zalet wymienionych wyżej rozwiązań należą:
\begin{itemize}
    \item \textbf{Szybkość i prostota}: Użytkownik otrzymuje wynik natychmiast po wypełnieniu krótkiej ankiety.
    \item \textbf{Dostępność}: Brak konieczności zakładania konta czy logowania.
    \item \textbf{Wiarygodność źródeł}: Bazują na ustandaryzowanych wytycznych medycznych.
\end{itemize}

Minusy tego formularza to:
\begin{itemize}
    \item \textbf{Brak personalizacji i wyjaśnienia}: Wynik jest liczbą lub statusem bez wyjaśnienia, który czynnik zaważył na diagnozie.
    \item \textbf{Brak historii}: Brak możliwości zapisu wyników i śledzenia zmian w czasie.
    \item Formularz na ekranie końcowym zniechęcający użytkowników do sprawdzenia wyników.
\end{itemize}

\subsection{Kalkulator platformy HaloDoctor}

Drugim analizowanym rozwiązaniem jest kalkulator ryzyka cukrzycy typu 2 dostępny na platformie telemedycznej HaloDoctor. Narzędzie to oparte jest na międzynarodowej skali FINDRISC (Finnish Diabetes Risk Score) i służy do oszacowania ryzyka zachorowania w perspektywie najbliższych 10 lat. Formularz składa się z ośmiu kluczowych pytań, które łączą parametry ilościowe (BMI, obwód talii) z wywiadem dotyczącym stylu życia (aktywność fizyczna, spożycie warzyw/owoców) oraz historią medyczną (nadciśnienie, hiperglikemia, obciążenie genetyczne).

\begin{figure}[htbp]
    \centering
    \includegraphics[width=0.8\textwidth]{figures/halodoctor.jpg}
    \caption{Kalkulator ryzyka HaloDoctor}
    \label{fig:halodoctor}
\end{figure}

Do głównych zalet tego rozwiązania należą:
\begin{itemize}
    \item \textbf{Integracja z usługami medycznymi}: Jako część platformy telemedycznej, narzędzie często sugeruje bezpośrednią konsultację lekarską w przypadku wysokiego wyniku, lub w celu prawdziwej diagnozy.
    \item \textbf{Przejrzysta skala punktowa}: Wynik jest sumowany w zakresie 0-24 punktów, co pozwala na łatwe przypisanie pacjenta do jednej z czterech grup ryzyka.
\end{itemize}

Minusy kalkulatora to:
\begin{itemize}
    \item \textbf{Subiektywność danych}: Pytania o dietę (np. "Czy jadasz warzywa codziennie?") opierają się na deklaracji użytkownika, co może prowadzić do zafałszowania wyniku.
    \item \textbf{Bariera wejścia}: Wymóg dokładnego zmierzenia obwodu w pasie może być problematyczny dla użytkownika korzystającego z aplikacji "w biegu".
    \item \textbf{Sztywny algorytm}: Podobnie jak w innych kalkulatorach opartych na FINDRISC, wagi poszczególnych pytań są stałe i nie dostosowują się dynamicznie do specyficznych kombinacji cech pacjenta.
\end{itemize}

\section{Wymagania funkcjonalne i niefunkcjonalne}

\subsection{Wymagania funkcjonalne}
\begin{itemize}
    \item Wprowadzanie danych diagnostycznych i metabolicznych (wiek, BMI, ciśnienie, cholesterol, poziom aktywności...) poprzez formularz.
    \item Wyświetlanie wyniku predykcji ryzyka wystąpienia cukrzycy (klasyfikacja: zdrowy, stan przedcukrzycowy, cukrzyca) w formie graficznej.
    \item Prezentacja interpretacji wyniku z wykorzystaniem wykresów SHAP wyjaśniających wpływ poszczególnych cech na decyzję modelu.
    \item Generowanie i wyświetlanie inteligentnych zaleceń zdrowotnych ("Smart Advisor") dopasowanych do wyniku użytkownika.
    \item Możliwość zapisywania codziennych logów zdrowotnych do bazy danych.
    \item Przeglądanie historii wykonanych predykcji oraz wprowadzonych logów.
    \item Wyświetlanie dashboardu analitycznego prezentującego trendy zmian ryzyka w czasie.
    \item Możliwość wykonania szybkiej symulacji ryzyka dla użytkownika niezalogowanego.
    \item Możliwość utworzenia nowego konta użytkownika.
    \item Możliwość zalogowania się w aplikacji.
    \item Możliwość usunięcia błędnych wpisów z dziennika zdrowia.
    \item Możliwość edycji profilu użytkownika (dane stałe, np. wzrost do obliczeń BMI).
\end{itemize}

\subsection{Wymagania niefunkcjonalne}
\begin{itemize}
    \item Responsywność w przeglądarkach internetowych na urządzeniach mobilnych i desktopowych.
    \item Wymagany stały dostęp do Internetu w celu komunikacji z API predykcyjnym.
    \item Czas odpowiedzi modelu predykcyjnego poniżej 1 sekundy.
    \item Skuteczność modelu uczenia maszynowego na poziomie minimum 80\%.
    \item Bezpieczne przechowywanie haseł oraz loginów użytkowników w bazie danych.
\end{itemize}

\section{Diagram przypadków użycia}



\subsection{Przypadek użycia: Szybka symulacja ryzyka}
\textbf{Aktor}: Gość \\
\textbf{Opis}: Wykonanie szybkiej analizy ryzyka cukrzycy bez konieczności logowania się. \\
\textbf{Warunki wstępne}: Użytkownik niezalogowany, otwarty ekran startowy aplikacji. \\
\textbf{Przebieg}:
\begin{enumerate}
    \item Gość wybiera opcję „Oblicz ryzyko cukrzycy”.
    \item System wyświetla formularz do diagnozy.
    \item Gość uzupełnia formularz i zatwierdza go.
    \item System przetwarza dane i wyświetla diagnozę.
    \item System wyświetla interpretację wyniku.
\end{enumerate}

\subsection{Przypadek użycia: Rejestracja}
\textbf{Aktor}: Gość \\
\textbf{Opis}: Utworzenie nowego konta użytkownika. \\
\textbf{Warunki wstępne}: Użytkownik nie posiada konta lub nie jest zalogowany. \\
\textbf{Przebieg}:
\begin{enumerate}
    \item Gość wybiera opcję rejestracji.
    \item System wyświetla formularz do rejestracji.
    \item Gość podaje wymagane dane i zatwierdza wybór.
    \item System weryfikuje poprawność danych i tworzy konto w bazie.
    \item Następuje automatyczne logowanie użytkownika.
\end{enumerate}

\subsection{Przypadek użycia: Logowanie}
\textbf{Aktor}: Gość \\
\textbf{Opis}: Uwierzytelnienie użytkownika w systemie. \\
\textbf{Warunki wstępne}: Użytkownik posiada konto, ale nie jest zalogowany. \\
\textbf{Przebieg}:
\begin{enumerate}
    \item Gość wybiera opcję Logowania.
    \item System wyświetla formularz logowania.
    \item Gość wprowadza login i hasło.
    \item System weryfikuje zgodność danych z bazą bezpiecznie przechowywanych haseł.
    \item[5a.] Dane są poprawne – użytkownik uzyskuje dostęp do aplikacji jako użytkownik zalogowany.
    \item[5b.] Dane są błędne – system wyświetla komunikat o błędzie.
    \item[6b.] Powrót do kroku 3.
\end{enumerate}

\subsection{Przypadek użycia: Wprowadzenie danych i predykcja ryzyka}
\textbf{Aktor}: Użytkownik Zalogowany \\
\textbf{Opis}: Wykonanie analizy ryzyka cukrzycy z trwałym zapisem wyniku w historii użytkownika w celu analizy trendów. \\
\textbf{Warunki wstępne}: Użytkownik zalogowany w aplikacji. \\
\textbf{Przebieg}:
\begin{enumerate}
    \item Użytkownik wybiera opcję „Oblicz ryzyko cukrzycy”.
    \item System wyświetla formularz do diagnozy.
    \item Użytkownik uzupełnia formularz i zatwierdza go.
    \item System przetwarza dane i wyświetla diagnozę.
    \item System wyświetla interpretację wyniku.
    \item System zapisuje dane w bazie danych.
\end{enumerate}

\subsection{Przypadek użycia: Przeglądanie historii i trendów}
\textbf{Aktor}: Użytkownik Zalogowany \\
\textbf{Opis}: Analiza zapisanych wyników oraz trendów zmian zdrowotnych w czasie. \\
\textbf{Warunki wstępne}: Użytkownik zalogowany, posiada w bazie co najmniej jeden zapisany wynik. \\
\textbf{Przebieg}:
\begin{enumerate}
    \item Użytkownik wybiera moduł analityczny.
    \item System pobiera historię badań zdrowotnych z bazy danych.
    \item System wyświetla wykresy trendów zmian ryzyka w czasie.
    \item Użytkownik może wybrać konkretny wpis, aby zobaczyć jego szczegóły.
\end{enumerate}

\subsection{Przypadek użycia: Zarządzanie dziennikiem zdrowia}
\textbf{Aktor}: Użytkownik Zalogowany \\
\textbf{Opis}: Edycja profilu oraz usuwanie błędnych wpisów z historii. \\
\textbf{Warunki wstępne}: Użytkownik zalogowany. \\
\textbf{Przebieg}:
\begin{enumerate}
    \item Użytkownik przegląda listę swoich wpisów zdrowotnych.
    \item Użytkownik wybiera błędny wpis i opcję usunięcia.
    \item System prosi o potwierdzenie i usuwa wpis z bazy danych.
    \item Użytkownik wchodzi w ustawienia profilu, aby zaktualizować dane stałe.
\end{enumerate}
