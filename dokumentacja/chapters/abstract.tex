\chapter*{Streszczenie}
\label{cha:streszczenie}
\makeatletter
\addcontentsline{toc}{section}{\textbf{Streszczenie}}

Niniejszy projekt, \textit{DiabetesRiskPredictor}, to nowoczesna aplikacja webowa służąca do przewidywania ryzyka wystąpienia cukrzycy typu 2. System integruje zaawansowane modele uczenia maszynowego (Logistic Regression, Random Forest, Gradient Boosting) z interaktywnym interfejsem użytkownika. Aplikacja nie tylko klasyfikuje ryzyko, ale również dostarcza spersonalizowanych wyjaśnień wyników dzięki wykorzystaniu metodologii SHAP (SHapley Additive exPlanations) oraz generatywnej sztucznej inteligencji (Google Gemini API).

Część serwerowa (Backend) została zaimplementowana w języku Python z użyciem frameworka Flask, natomiast część kliencka (Frontend) wykorzystuje bibliotekę React. Projekt kładzie duży nacisk na wyjaśnialność (XAI) oraz edukację użytkownika w zakresie profilaktyki zdrowotnej.