\chapter{Przegląd Aktualnych Rozwiązań}
\label{cha:przeglad}

\section{Formularz American Diabetes Association}

Internetowy kalkulatory ryzyka cukrzycy - formularz American Diabetes Association jest jednym z najczęściej spotykanych rozwiązań pozwalających na szybką samoocenę stanu zdrowia. Umożliwiają one wprowadzenie podstawowych biometrycznych, takich jak wiek, waga czy wzrost, w celu oszacowania ryzyka zachorowania w oparciu o sztywne reguły punktowe.

\begin{figure}[htbp]
    \centering
    \begin{subfigure}[b]{0.48\textwidth}
        \centering
        \includegraphics[width=\textwidth]{figures/formularzamericanda1.jpg}
        \caption{Część 1}
    \end{subfigure}
    \hfill
    \begin{subfigure}[b]{0.48\textwidth}
        \centering
        \includegraphics[width=\textwidth]{figures/formularzamericanda2.jpg}
        \caption{Część 2}
    \end{subfigure}
    \caption{Formularz American Diabetes Association}
    \label{fig:ada_form}
\end{figure}

Do głównych zalet wymienionych wyżej rozwiązań należą:
\begin{itemize}
    \item \textbf{Szybkość i prostota}: Użytkownik otrzymuje wynik natychmiast po wypełnieniu krótkiej ankiety.
    \item \textbf{Dostępność}: Brak konieczności zakładania konta czy logowania.
    \item \textbf{Wiarygodność źródeł}: Bazują na ustandaryzowanych wytycznych medycznych.
\end{itemize}

Minusy tego formularza to:
\begin{itemize}
    \item \textbf{Brak personalizacji i wyjaśnienia}: Wynik jest liczbą lub statusem bez wyjaśnienia, który czynnik zaważył na diagnozie.
    \item \textbf{Brak historii}: Brak możliwości zapisu wyników i śledzenia zmian w czasie.
    \item Formularz na ekranie końcowym zniechęcający użytkowników do sprawdzenia wyników.
\end{itemize}

\section{Kalkulator platformy HaloDoctor}

Drugim analizowanym rozwiązaniem jest kalkulator ryzyka cukrzycy typu 2 dostępny na platformie telemedycznej HaloDoctor. Narzędzie to oparte jest na międzynarodowej skali FINDRISC (Finnish Diabetes Risk Score) i służy do oszacowania ryzyka zachorowania w perspektywie najbliższych 10 lat. Formularz składa się z ośmiu kluczowych pytań, które łączą parametry ilościowe (BMI, obwód talii) z wywiadem dotyczącym stylu życia (aktywność fizyczna, spożycie warzyw/owoców) oraz historią medyczną (nadciśnienie, hiperglikemia, obciążenie genetyczne).

\begin{figure}[htbp]
    \centering
    \includegraphics[width=0.8\textwidth]{figures/halodoctor.jpg}
    \caption{Kalkulator ryzyka HaloDoctor}
    \label{fig:halodoctor}
\end{figure}

Do głównych zalet tego rozwiązania należą:
\begin{itemize}
    \item \textbf{Integracja z usługami medycznymi}: Jako część platformy telemedycznej, narzędzie często sugeruje bezpośrednią konsultację lekarską w przypadku wysokiego wyniku, lub w celu prawdziwej diagnozy.
    \item \textbf{Przejrzysta skala punktowa}: Wynik jest sumowany w zakresie 0-24 punktów, co pozwala na łatwe przypisanie pacjenta do jednej z czterech grup ryzyka.
\end{itemize}

Minusy kalkulatora to:
\begin{itemize}
    \item \textbf{Subiektywność danych}: Pytania o dietę (np. "Czy jadasz warzywa codziennie?") opierają się na deklaracji użytkownika, co może prowadzić do zafałszowania wyniku.
    \item \textbf{Bariera wejścia}: Wymóg dokładnego zmierzenia obwodu w pasie może być problematyczny dla użytkownika korzystającego z aplikacji "w biegu".
    \item \textbf{Sztywny algorytm}: Podobnie jak w innych kalkulatorach opartych na FINDRISC, wagi poszczególnych pytań są stałe i nie dostosowują się dynamicznie do specyficznych kombinacji cech pacjenta.
\end{itemize}
