\chapter{Wstęp i Architektura Systemu}
\label{cha:wstep}

\section{Wprowadzenie}
Cukrzyca typu 2 jest jedną z najszybciej rozwijających się chorób cywilizacyjnych. Wczesna diagnoza i świadomość czynników ryzyka są kluczowe dla skutecznej prewencji. Celem projektu \textit{DiabetesRiskPredictor} jest stworzenie dostępnego narzędzia, które na podstawie danych ankietowych użytkownika (takich jak BMI, aktywność fizyczna, nawyki żywieniowe) oszacuje ryzyko zachorowania oraz dostarczy zrozumiałych rekomendacji zdrowotnych.

\section{Architektura Systemu}
System został zaprojektowany w architekturze klient-serwer, z wyraźnym podziałem na warstwę prezentacji, logiki biznesowej oraz modeli predykcyjnych.

\begin{figure}[htbp]
    \centering
\begin{verbatim}
+-----------------+         +-----------------+         +-----------------+
|    Frontend     |  HTTP   |     Backend     |         |   ML Models     |
|  React + Vite   | <-----> |  Flask + SQLite | <-----> |  Scikit-learn   |
|   Port: 5173    |         |   Port: 5000    |         |   + SHAP        |
+-----------------+         +-----------------+         +-----------------+
                                    |
                                    v
                            +-----------------+
                            |  Google Gemini  |
                            |   (AI Advice)   |
                            +-----------------+
    \end{verbatim}
    \caption{Schemat architektury systemu DiabetesRiskPredictor}
    \label{fig:architektura}
\end{figure}

Główne komponenty systemu to:
\begin{itemize}
    \item \textbf{Frontend}: Aplikacja typu Single Page Application (SPA) zbudowana w oparciu o React 19 i Vite. Odpowiada za interakcję z użytkownikiem, zbieranie danych ankietowych oraz prezentację wyników.
    \item \textbf{Backend}: Serwer API napisany w Pythonie (Flask). Obsługuje żądania HTTP, zarządza bazą danych użytkowników, autentykacją (JWT) oraz komunikacją z modelami ML.
    \item \textbf{Modele ML}: Zestaw wytrenowanych modeli klasyfikacyjnych (Scikit-learn) służących do oceny ryzyka.
    \item \textbf{Moduł AI}: Integracja z Google Gemini API w celu generowania spersonalizowanych porad zdrowotnych w języku naturalnym.
\end{itemize}

\section{Stos Technologiczny}

W tabeli \ref{tab:technologie} przedstawiono szczegółowy stos technologiczny wykorzystany w projekcie.

\begin{table}[htbp]
    \centering
    \caption{Technologie wykorzystane w projekcie}
    \label{tab:technologie}
    \begin{tabular}{|l|l|}
        \hline
        \textbf{Warstwa} & \textbf{Technologia} \\
        \hline
        Frontend & React 19, TypeScript, Vite, TailwindCSS (opcjonalnie) \\
        \hline
        Backend & Python 3, Flask, SQLAlchemy \\
        \hline
        Machine Learning & Scikit-learn, Pandas, NumPy, SHAP \\
        \hline
        Sztuczna Inteligencja & Google Gemini API \\
        \hline
        Autentykacja & JWT (JSON Web Tokens) \\
        \hline
        Baza Danych & SQLite (rozwojowa) / PostgreSQL (produkcyjna) \\
        \hline
    \end{tabular}
\end{table}