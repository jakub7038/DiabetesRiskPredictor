\chapter{Implementacja i API}
\label{cha:implementacja}

\section{Backend (Flask)}
Backend aplikacji pełni rolę huba integrującego bazę danych, modele ML oraz zewnętrzne API. Został zrealizowany w mikro-frameworku Flask, co zapewnia lekkość i łatwość rozbudowy.

\subsection{Struktura API}

Główne endpointy API to:

\begin{itemize}
    \item \texttt{POST /api/predict} - Główny endpoint. Przyjmuje JSON z danymi ankiety, zwraca wyniki ze wszystkich 3 modeli, analizę SHAP oraz poradę AI.
    \item \texttt{POST /api/auth/register} - Rejestracja nowego użytkownika.
    \item \texttt{POST /api/auth/login} - Logowanie (zwraca token JWT access + refresh).
    \item \texttt{GET /api/user/history} - Pobranie historii predykcji zalogowanego użytkownika.
\end{itemize}

\subsection{Model Danych (SQLAlchemy)}
Baza danych przechowuje informacje o użytkownikach oraz historię ich badań. Wykorzystano ORM SQLAlchemy.

Relacje:
\begin{itemize}
    \item \texttt{User} (1) $\longleftrightarrow$ (N) \texttt{PredictionResult}
    \item \texttt{User} (1) $\longleftrightarrow$ (1) \texttt{UserProfile} (dane demograficzne)
\end{itemize}

\section{Frontend (React)}
Warstwa prezentacji została zbudowana jako Single Page Application. Wykorzystuje:
\begin{itemize}
    \item \textbf{React Router}: Do obsługi nawigacji bez przeładowywania strony.
    \item \textbf{Axios}: Do komunikacji z API.
    \item \textbf{Context API}: Do zarządzania stanem sesji użytkownika (autentykacja).
\end{itemize}

Proces badania składa się z 3 kroków (formularz wieloetapowy):
1. Podstawowe dane (Wiek, Płeć).
2. Parametry zdrowotne (BMI, Nadciśnienie).
3. Styl życia (Dieta, Używki).

Poniżej przedstawiono fragment komponentu formularza obsługującego wysyłkę danych:

\begin{lstlisting}[style=javaStyle, caption=Obsluga formularza w React, label=lst:react]
const handleSubmit = async (formData) => {
    try {
        const response = await api.post('/predict', formData);
        setResult(response.data);
        navigate('/results');
    } catch (error) {
        console.error("Blad predykcji:", error);
    }
};
\end{lstlisting}

\subsection{Widoki Aplikacji}
Interfejs użytkownika został zaprojektowany z myślą o przejrzystości i łatwości obsługi. Poniżej przedstawiono kluczowe ekrany aplikacji.

\subsubsection{Strona Główna i Formularz}
Strona główna wita użytkownika i wyjaśnia cel aplikacji. Głównym elementem jest wieloetapowy formularz (Wizard), który prowadzi pacjenta przez proces wprowadzania danych.

\begin{figure}[H]
    \centering
    \includegraphics[width=1.0\textwidth]{figures/landingpage.png}
    \caption{Interfejs strony głównej}
    \label{fig:home}
\end{figure}

Kroki formularza zostały zaprojektowane tak, aby były intuicyjne i nie przytłaczały ilością pytań na jednym ekranie.

\begin{figure}[H]
    \centering
    \includegraphics[width=1.0\textwidth]{figures/formularz.png}
    \caption{Widok kroków formularza diagnostycznego}
    \label{fig:form_steps}
\end{figure}

\subsubsection{Prezentacja Wyników}
Po przetworzeniu danych, użytkownik otrzymuje czytelny wynik w formie graficznej (wskaźnik ryzyka) oraz tekstowej. Kolorystyka (zielony/żółty/czerwony) natychmiastowo informuje o poziomie zagrożenia.

\begin{figure}[H]
    \centering
    \includegraphics[width=1.0\textwidth]{figures/wynikguest.png}
    \caption{Ekran wyników predykcji z wykresem ryzyka}
    \label{fig:results}
\end{figure}

\subsubsection{Analiza Szczegółowa (SHAP i AI)}
Dla bardziej dociekliwych użytkowników dostępna jest sekcja szczegółowa, zawierająca spersonalizowaną poradę wygenerowaną przez asystenta AI.

\begin{figure}[H]
    \centering
    \includegraphics[width=1.0\textwidth]{figures/wynikiuserloggewdin.png}
    \caption{Szczegółowa analiza czynników ryzyka i porada AI}
    \label{fig:shap_ai}
\end{figure}

\subsubsection{Panel Użytkownika i Historia}
Zalogowani użytkownicy mają dostęp do panelu, w którym mogą zarządzać swoim profilem oraz przeglądać historię wykonanych badań.

\begin{figure}[H]
    \centering
    \includegraphics[width=1.0\textwidth]{figures/userpage.png}
    \caption{Panel użytkownika}
    \label{fig:user_panel}
\end{figure}

Sekcja historii pozwala na śledzenie zmian ryzyka w czasie, co jest kluczowe dla prewencji długoterminowej.

\begin{figure}[H]
    \centering
    \includegraphics[width=1.0\textwidth]{figures/historiapredykcji.png}
    \caption{Widok historii predykcji}
    \label{fig:history}
\end{figure}

\section{Podsumowanie}
Stworzony system \textit{DiabetesRiskPredictor} stanowi kompleksowe rozwiązanie demonstrujące praktyczne zastosowanie uczenia maszynowego w medycynie prewencyjnej. Połączenie klasycznych modeli klasyfikacyjnych z nowoczesną generatywną AI pozwala na dostarczenie użytkownikowi wartościowej i zrozumiałej informacji zwrotnej.
