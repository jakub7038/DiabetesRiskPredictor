\chapter{Analiza Wymagań}
\label{cha:wymagania}

\section{Wymagania funkcjonalne i niefunkcjonalne}

\subsection{Wymagania funkcjonalne}
\begin{itemize}
    \item Wprowadzanie danych diagnostycznych i metabolicznych (wiek, BMI, ciśnienie, cholesterol, poziom aktywności...) poprzez formularz.
    \item Wyświetlanie wyniku predykcji ryzyka wystąpienia cukrzycy (klasyfikacja: zdrowy, stan przedcukrzycowy, cukrzyca) w formie graficznej.
    \item Generowanie i wyświetlanie inteligentnych zaleceń zdrowotnych ("Smart Advisor") dopasowanych do wyniku użytkownika.
    \item Przeglądanie historii wykonanych predykcji oraz wprowadzonych logów.
    \item Możliwość wykonania szybkiej symulacji ryzyka dla użytkownika niezalogowanego.
    \item Możliwość utworzenia nowego konta użytkownika.
    \item Możliwość zalogowania się w aplikacji.
    \item Możliwość usunięcia błędnych wpisów z dziennika zdrowia.
    \item Możliwość edycji profilu użytkownika (dane stałe, np. wzrost do obliczeń BMI).
\end{itemize}

\subsection{Wymagania niefunkcjonalne}
\begin{itemize}
    \item Wymagany stały dostęp do Internetu w celu komunikacji z API predykcyjnym.
    \item Czas odpowiedzi modelu predykcyjnego poniżej 1 sekundy.
    \item Skuteczność modelu uczenia maszynowego na poziomie minimum 80\%.
    \item Bezpieczne przechowywanie haseł oraz loginów użytkowników w bazie danych.
\end{itemize}

\section{Diagram przypadków użycia}


\subsection{Przypadek użycia: Szybka symulacja ryzyka}
\textbf{Aktor}: Gość \\
\textbf{Opis}: Wykonanie szybkiej analizy ryzyka cukrzycy bez konieczności logowania się. \\
\textbf{Warunki wstępne}: Użytkownik niezalogowany, otwarty ekran startowy aplikacji. \\
\textbf{Przebieg}:
\begin{enumerate}
    \item Gość wybiera opcję „Oblicz ryzyko cukrzycy”.
    \item System wyświetla formularz do diagnozy.
    \item Gość uzupełnia formularz i zatwierdza go.
    \item System przetwarza dane i wyświetla diagnozę.
    \item System wyświetla interpretację wyniku.
\end{enumerate}

\subsection{Przypadek użycia: Rejestracja}
\textbf{Aktor}: Gość \\
\textbf{Opis}: Utworzenie nowego konta użytkownika. \\
\textbf{Warunki wstępne}: Użytkownik nie posiada konta lub nie jest zalogowany. \\
\textbf{Przebieg}:
\begin{enumerate}
    \item Gość wybiera opcję rejestracji.
    \item System wyświetla formularz do rejestracji.
    \item Gość podaje wymagane dane i zatwierdza wybór.
    \item System weryfikuje poprawność danych i tworzy konto w bazie.
    \item Następuje automatyczne logowanie użytkownika.
\end{enumerate}

\subsection{Przypadek użycia: Logowanie}
\textbf{Aktor}: Gość \\
\textbf{Opis}: Uwierzytelnienie użytkownika w systemie. \\
\textbf{Warunki wstępne}: Użytkownik posiada konto, ale nie jest zalogowany. \\
\textbf{Przebieg}:
\begin{enumerate}
    \item Gość wybiera opcję Logowania.
    \item System wyświetla formularz logowania.
    \item Gość wprowadza login i hasło.
    \item System weryfikuje zgodność danych z bazą bezpiecznie przechowywanych haseł.
    \item[5a.] Dane są poprawne – użytkownik uzyskuje dostęp do aplikacji jako użytkownik zalogowany.
    \item[5b.] Dane są błędne – system wyświetla komunikat o błędzie.
    \item[6b.] Powrót do kroku 3.
\end{enumerate}

\subsection{Przypadek użycia: Symulacja ryzyka oraz porady}
\textbf{Aktor}: Użytkownik Zalogowany \\
\textbf{Opis}: Wykonanie analizy ryzyka cukrzycy z trwałym zapisem wyniku w historii użytkownika. \\
\textbf{Warunki wstępne}: Użytkownik zalogowany w aplikacji. \\
\textbf{Przebieg}:
\begin{enumerate}
    \item Użytkownik wybiera opcję „Oblicz ryzyko cukrzycy”.
    \item System wyświetla formularz do diagnozy.
    \item Użytkownik uzupełnia formularz i zatwierdza go.
    \item System przetwarza dane i wyświetla diagnozę wraz z poradami.
    \item System zapisuje dane w bazie danych.
\end{enumerate}

\subsection{Przypadek użycia: Przeglądanie historii predykcji}
\textbf{Aktor}: Użytkownik Zalogowany \\
\textbf{Opis}: Przeglądanie zapisanych wyników badań. \\
\textbf{Warunki wstępne}: Użytkownik zalogowany, posiada w bazie co najmniej jeden zapisany wynik. \\
\textbf{Przebieg}:
\begin{enumerate}
    \item Użytkownik wybiera moduł historii.
    \item System pobiera historię badań zdrowotnych z bazy danych.
    \item System wyświetla listę wykonanych predykcji.
    \item Użytkownik może wybrać konkretny wpis, aby zobaczyć jego szczegóły.
\end{enumerate}
