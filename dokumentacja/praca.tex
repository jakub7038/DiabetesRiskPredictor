%          
% Author:  Zbigniew Gomółka     (zgomolka@ur.edu.pl)
%          Ewa Żesławska        (ezeslawska@ur.edu.pl)
%          Uniwersystet Rzeszowski, Rzeszów, POLAND
% Aleksander Wojtowicz
% Marcin Mrukowicz

\documentclass{urdpl}     % praca w języku polskim


% Lista wszystkich języków stanowiących języki pozycji bibliograficznych użytych w pracy.
% (Zgodnie z zasadami tworzenia bibliografii każda pozycja powinna zostać utworzona zgodnie z zasadami języka, w którym dana publikacja została napisana.)
\usepackage[english,polish]{babel}


% Użyj polskiego łamania wyrazów (zamiast domyślnego angielskiego).
\usepackage{polski}
\usepackage[utf8]{inputenc}
% dodatkowe pakiety
\usepackage{mathtools}
\usepackage{amsfonts}
\usepackage{amsmath}
\usepackage{amsthm}
\usepackage[hidelinks]{hyperref}
\usepackage{float}
\usepackage{listings}
\usepackage{graphicx}
\usepackage{subcaption}
\usepackage{booktabs}
\usepackage{multirow} 
\usepackage{tabularx} 
\usepackage{amssymb} 
\usepackage{listings}
\usepackage{xcolor}
\usepackage{array}
\usepackage{makecell}
\usepackage[flushleft]{threeparttable}
\usepackage[normalem]{ulem}
\usepackage{lineno}
\newcolumntype{L}{>{\raggedright\arraybackslash}X} % left-wrapped
\newcolumntype{C}{>{\centering\arraybackslash}X}   % centered-wrapped
\usepackage{ltablex}  
\usepackage{tocloft}     % For creating custom lists
\keepXColumns
% definicja dodatkowych typów kolumn
\newcolumntype{Y}{>{\raggedright\arraybackslash}X} % elastyczna kolumna, łamana w lewo
\newcolumntype{R}{>{\raggedleft\arraybackslash}X}  % elastyczna kolumna do prawej

% ---------------------------------------------

% --- < bibliografia > ---

\usepackage{csquotes}


% ------------------------
% --- < listingi > ---

% Użyj czcionki kroju Courier.
\usepackage{courier}

\usepackage{listings}
\lstloadlanguages{TeX}
\renewcommand{\lstlistlistingname}{Spis listingów}
\renewcommand{\lstlistingname}{Listing}

\lstset{
	literate={ą}{{\k{a}}}1
           {ć}{{\'c}}1
           {ę}{{\k{e}}}1
           {ó}{{\'o}}1
           {ń}{{\'n}}1
           {ł}{{\l{}}}1
           {ś}{{\'s}}1
           {ź}{{\'z}}1
           {ż}{{\.z}}1
           {Ą}{{\k{A}}}1
           {Ć}{{\'C}}1
           {Ę}{{\k{E}}}1
           {Ó}{{\'O}}1
           {Ń}{{\'N}}1
           {Ł}{{\L{}}}1
           {Ś}{{\'S}}1
           {Ź}{{\'Z}}1
           {Ż}{{\.Z}}1,
	basicstyle=\footnotesize\ttfamily,
}

% defninicja stylu python
\lstdefinestyle{stylePython}{
    language=Python,
    commentstyle=\color{green},          % Kolor komentarzy
    keywordstyle=\color{blue},           % Kolor słów kluczowych
    numberstyle=\tiny\color{gray},       % Kolor i styl numerów linii
    stringstyle=\color{red},             % Kolor ciągów znaków
    basicstyle=\ttfamily\footnotesize,   % Podstawowy styl kodu
    breakatwhitespace=false,             % Automatyczne dzielenie wierszy
    breaklines=true,                     % Dzielenie długich linii
    keepspaces=true,                     % Zachowanie spacji
    numbers=left,                        % Numery linii po lewej
    numbersep=5pt,                       % Odstęp numerów od kodu
    showspaces=false,                    % Nie pokazuj spacji
    showstringspaces=false,              % Nie pokazuj spacji w ciągach znaków
    showtabs=false,                      % Nie pokazuj tabulacji
    tabsize=2                            % Rozmiar tabulacji
}

% defnicja stylu JAVA
\lstdefinestyle{javaStyle}{
    language=Java,
    basicstyle=\ttfamily\footnotesize,
    keywordstyle=\color{blue},
    commentstyle=\color{green!50!black}\itshape,
    stringstyle=\color{green},
    numberstyle=\tiny\color{gray},
    numbers=left,
    numbersep=5pt,                       % Odstęp numerów od kodu
    stepnumber=1,
    showspaces=false,                    % Nie pokazuj spacji
    tabsize=2,
    showstringspaces=false,
    breaklines=true,
    breakatwhitespace=false,             % Automatyczne dzielenie wierszy
    showtabs=false,                      % Nie pokazuj tabulacji
    keepspaces=true                      % Zachowanie spacji
}


\definecolor{stringcolor}{RGB}{163,21,21}    % pomarańczowy - string
\definecolor{typecolor}{RGB}{43, 145, 176}     % ciemny fiolet - klasy, typy

\lstdefinestyle{csStyle}{
    language=[Sharp]C,                      % dla C#; można zmienić na Java
    basicstyle=\ttfamily\footnotesize,
    keywordstyle=\color{blue},
    stringstyle=\color{stringcolor},
    commentstyle=\color{green!50!black}\itshape,
    morekeywords={class, public, private, protected, static, void, string, int, new}, % dodatkowe słowa kluczowe
    emphstyle=\color{typecolor}\bfseries, % klasy na fioletowo
    numbers=left,
    numbersep=5pt,                       % Odstęp numerów od kodu
    numberstyle=\tiny\color{gray},
    stepnumber=1,
    breaklines=true,
    showspaces=false,                    % Nie pokazuj spacji
    tabsize=2,
    showstringspaces=false,
    breakatwhitespace=false,             % Automatyczne dzielenie wierszy
    showtabs=false,                      % Nie pokazuj tabulacji
    keepspaces=true                      % Zachowanie spacji  
}

\definecolor{lightgray}{rgb}{0.9,0.9,0.9}
    \definecolor{green}{rgb}{0,0.6,0}
    \definecolor{gray}{rgb}{0.5,0.5,0.5}

% % ------------------------
\AtBeginDocument{
	\renewcommand{\tablename}{Tabela}
	\renewcommand{\figurename}{Rys.}   
    \newcommand{\listingname}{Listing}
}


% ----------------------------------------------------
% Tworzymy niezależny float prompt
% ----------------------------------------------------
\newfloat{prompt}{htbp}{lop}[section]
\floatname{prompt}{Prompt}

% Tworzymy listę promptów na wzór listy rysunków
\newcommand{\listofprompts}{\listof{prompt}{Spis promptów}}

% ----------------------------------------------------
% Nowe polecenie: podpis do promptu - alias zwykłego caption
% ----------------------------------------------------
\newcommand{\promptcaption}[1]{%
  \caption{#1}%
}

% ----------------------------------------------------
% Opcjonalnie można wykorzystać skrócone polecenie
% ----------------------------------------------------
\newcommand{\promptfig}[3][]{%
\begin{prompt}[htbp]
    \centering
    \includegraphics[#1]{#2}
    \promptcaption{#3}
\end{prompt}
}


% ------------------------
% --- < tabele > ---

% defines the X column to use m (\parbox[c]) instead of p (`parbox[t]`)
\newcolumntype{CY}[1]{>{\hsize=#1\hsize\centering\arraybackslash}X}

%---------------------------------------------------------------------------

\author{Jakub Siłka \\ Karol Brudniak \\ Paweł Powęska}
\shortauthor{J. Siłka, K. Brudniak, P. Powęska}
\noAlbum{131509, 125106, 131496}

\titlePL{System predykcji ryzyka cukrzycy typu 2 z wykorzystaniem uczenia maszynowego i AI}
\titleEN{Diabetes Risk Prediction System using Machine Learning and AI}

\shorttitlePL{System predykcji ryzyka cukrzycy}
\shorttitleEN{Diabetes Risk Prediction System}

\thesistype{Projekt Inżynierski}

\thesisDone{Praca wykonana pod kierunkiem}
\supervisor{mgr inż. Wojciech Gałka}
%\supervisor{Jan Nowak PhD}

\degreeprogramme{Informatyka}
%\degreeprogramme{Computer Science}

\date{2026}

\department{Instytut Informatyki}
%\department{Institute of Computer Science}

\faculty{Wydział Nauk Ścisłych i Technicznych}
%\faculty{Faculty of Science and Technology}



\setlength{\cftsecnumwidth}{10mm}

%---------------------------------------------------------------------------
\setcounter{secnumdepth}{4}
\brokenpenalty=10000\relax

% --------------------------------------------------------------------------
% główna część pracy
% --------------------------------------------------------------------------

\begin{document}

\titlepages

% Ponowne zdefiniowanie stylu `plain`, aby usunąć numer strony z pierwszej strony spisu treści i poszczególnych rozdziałów.
\fancypagestyle{plain}
{
    % Usuń nagłówek i stopkę
    \fancyhf{}
    % Usuń linie.
    \renewcommand{\headrulewidth}{0pt}
    \renewcommand{\footrulewidth}{0pt}
}

\setcounter{tocdepth}{2}
\tableofcontents
\clearpage


% dodanie poszczególnych rozdziałów 
\chapter{Wstęp i Architektura Systemu}
\label{cha:wstep}

\section{Wprowadzenie}
Cukrzyca typu 2 jest jedną z najszybciej rozwijających się chorób cywilizacyjnych. Wczesna diagnoza i świadomość czynników ryzyka są kluczowe dla skutecznej prewencji. Celem projektu \textit{DiabetesRiskPredictor} jest stworzenie dostępnego narzędzia, które na podstawie danych ankietowych użytkownika (takich jak BMI, aktywność fizyczna, nawyki żywieniowe) oszacuje ryzyko zachorowania oraz dostarczy zrozumiałych rekomendacji zdrowotnych.

\section{Architektura Systemu}
System został zaprojektowany w architekturze klient-serwer, z wyraźnym podziałem na warstwę prezentacji, logiki biznesowej oraz modeli predykcyjnych.

\begin{figure}[htbp]
    \centering
\begin{verbatim}
+-----------------+         +-----------------+         +-----------------+
|    Frontend     |  HTTP   |     Backend     |         |   ML Models     |
|  React + Vite   | <-----> |  Flask + SQLite | <-----> |  Scikit-learn   |
|   Port: 5173    |         |   Port: 5000    |         |   + SHAP        |
+-----------------+         +-----------------+         +-----------------+
                                    |
                                    v
                            +-----------------+
                            |  Google Gemini  |
                            |   (AI Advice)   |
                            +-----------------+
    \end{verbatim}
    \caption{Schemat architektury systemu DiabetesRiskPredictor}
    \label{fig:architektura}
\end{figure}

Główne komponenty systemu to:
\begin{itemize}
    \item \textbf{Frontend}: Aplikacja typu Single Page Application (SPA) zbudowana w oparciu o React 19 i Vite. Odpowiada za interakcję z użytkownikiem, zbieranie danych ankietowych oraz prezentację wyników.
    \item \textbf{Backend}: Serwer API napisany w Pythonie (Flask). Obsługuje żądania HTTP, zarządza bazą danych użytkowników, autentykacją (JWT) oraz komunikacją z modelami ML.
    \item \textbf{Modele ML}: Zestaw wytrenowanych modeli klasyfikacyjnych (Scikit-learn) służących do oceny ryzyka.
    \item \textbf{Moduł AI}: Integracja z Google Gemini API w celu generowania spersonalizowanych porad zdrowotnych w języku naturalnym.
\end{itemize}

\section{Stos Technologiczny}

W tabeli \ref{tab:technologie} przedstawiono szczegółowy stos technologiczny wykorzystany w projekcie.

\begin{table}[htbp]
    \centering
    \caption{Technologie wykorzystane w projekcie}
    \label{tab:technologie}
    \begin{tabular}{|l|l|}
        \hline
        \textbf{Warstwa} & \textbf{Technologia} \\
        \hline
        Frontend & React 19, TypeScript, Vite, TailwindCSS (opcjonalnie) \\
        \hline
        Backend & Python 3, Flask, SQLAlchemy \\
        \hline
        Machine Learning & Scikit-learn, Pandas, NumPy, SHAP \\
        \hline
        Sztuczna Inteligencja & Google Gemini API \\
        \hline
        Autentykacja & JWT (JSON Web Tokens) \\
        \hline
        Baza Danych & SQLite (rozwojowa) / PostgreSQL (produkcyjna) \\
        \hline
    \end{tabular}
\end{table}
\chapter{Przegląd Aktualnych Rozwiązań}
\label{cha:przeglad}

\section{Formularz American Diabetes Association}

Internetowy kalkulatory ryzyka cukrzycy - formularz American Diabetes Association jest jednym z najczęściej spotykanych rozwiązań pozwalających na szybką samoocenę stanu zdrowia. Umożliwiają one wprowadzenie podstawowych biometrycznych, takich jak wiek, waga czy wzrost, w celu oszacowania ryzyka zachorowania w oparciu o sztywne reguły punktowe.

\begin{figure}[htbp]
    \centering
    \begin{subfigure}[b]{0.48\textwidth}
        \centering
        \includegraphics[width=\textwidth]{figures/formularzamericanda1.jpg}
        \caption{Część 1}
    \end{subfigure}
    \hfill
    \begin{subfigure}[b]{0.48\textwidth}
        \centering
        \includegraphics[width=\textwidth]{figures/formularzamericanda2.jpg}
        \caption{Część 2}
    \end{subfigure}
    \caption{Formularz American Diabetes Association}
    \label{fig:ada_form}
\end{figure}

Do głównych zalet wymienionych wyżej rozwiązań należą:
\begin{itemize}
    \item \textbf{Szybkość i prostota}: Użytkownik otrzymuje wynik natychmiast po wypełnieniu krótkiej ankiety.
    \item \textbf{Dostępność}: Brak konieczności zakładania konta czy logowania.
    \item \textbf{Wiarygodność źródeł}: Bazują na ustandaryzowanych wytycznych medycznych.
\end{itemize}

Minusy tego formularza to:
\begin{itemize}
    \item \textbf{Brak personalizacji i wyjaśnienia}: Wynik jest liczbą lub statusem bez wyjaśnienia, który czynnik zaważył na diagnozie.
    \item \textbf{Brak historii}: Brak możliwości zapisu wyników i śledzenia zmian w czasie.
    \item Formularz na ekranie końcowym zniechęcający użytkowników do sprawdzenia wyników.
\end{itemize}

\section{Kalkulator platformy HaloDoctor}

Drugim analizowanym rozwiązaniem jest kalkulator ryzyka cukrzycy typu 2 dostępny na platformie telemedycznej HaloDoctor. Narzędzie to oparte jest na międzynarodowej skali FINDRISC (Finnish Diabetes Risk Score) i służy do oszacowania ryzyka zachorowania w perspektywie najbliższych 10 lat. Formularz składa się z ośmiu kluczowych pytań, które łączą parametry ilościowe (BMI, obwód talii) z wywiadem dotyczącym stylu życia (aktywność fizyczna, spożycie warzyw/owoców) oraz historią medyczną (nadciśnienie, hiperglikemia, obciążenie genetyczne).

\begin{figure}[htbp]
    \centering
    \includegraphics[width=0.8\textwidth]{figures/halodoctor.jpg}
    \caption{Kalkulator ryzyka HaloDoctor}
    \label{fig:halodoctor}
\end{figure}

Do głównych zalet tego rozwiązania należą:
\begin{itemize}
    \item \textbf{Integracja z usługami medycznymi}: Jako część platformy telemedycznej, narzędzie często sugeruje bezpośrednią konsultację lekarską w przypadku wysokiego wyniku, lub w celu prawdziwej diagnozy.
    \item \textbf{Przejrzysta skala punktowa}: Wynik jest sumowany w zakresie 0-24 punktów, co pozwala na łatwe przypisanie pacjenta do jednej z czterech grup ryzyka.
\end{itemize}

Minusy kalkulatora to:
\begin{itemize}
    \item \textbf{Subiektywność danych}: Pytania o dietę (np. "Czy jadasz warzywa codziennie?") opierają się na deklaracji użytkownika, co może prowadzić do zafałszowania wyniku.
    \item \textbf{Bariera wejścia}: Wymóg dokładnego zmierzenia obwodu w pasie może być problematyczny dla użytkownika korzystającego z aplikacji "w biegu".
    \item \textbf{Sztywny algorytm}: Podobnie jak w innych kalkulatorach opartych na FINDRISC, wagi poszczególnych pytań są stałe i nie dostosowują się dynamicznie do specyficznych kombinacji cech pacjenta.
\end{itemize}

\chapter{Analiza Wymagań}
\label{cha:wymagania}

\section{Wymagania funkcjonalne i niefunkcjonalne}

\subsection{Wymagania funkcjonalne}
\begin{itemize}
    \item Wprowadzanie danych diagnostycznych i metabolicznych (wiek, BMI, ciśnienie, cholesterol, poziom aktywności...) poprzez formularz.
    \item Wyświetlanie wyniku predykcji ryzyka wystąpienia cukrzycy (klasyfikacja: zdrowy, stan przedcukrzycowy, cukrzyca) w formie graficznej.
    \item Generowanie i wyświetlanie inteligentnych zaleceń zdrowotnych ("Smart Advisor") dopasowanych do wyniku użytkownika.
    \item Przeglądanie historii wykonanych predykcji oraz wprowadzonych logów.
    \item Możliwość wykonania szybkiej symulacji ryzyka dla użytkownika niezalogowanego.
    \item Możliwość utworzenia nowego konta użytkownika.
    \item Możliwość zalogowania się w aplikacji.
    \item Możliwość usunięcia błędnych wpisów z dziennika zdrowia.
    \item Możliwość edycji profilu użytkownika (dane stałe, np. wzrost do obliczeń BMI).
\end{itemize}

\subsection{Wymagania niefunkcjonalne}
\begin{itemize}
    \item Wymagany stały dostęp do Internetu w celu komunikacji z API predykcyjnym.
    \item Czas odpowiedzi modelu predykcyjnego poniżej 1 sekundy.
    \item Skuteczność modelu uczenia maszynowego na poziomie minimum 80\%.
    \item Bezpieczne przechowywanie haseł oraz loginów użytkowników w bazie danych.
\end{itemize}

\section{Diagram przypadków użycia}


\subsection{Przypadek użycia: Szybka symulacja ryzyka}
\textbf{Aktor}: Gość \\
\textbf{Opis}: Wykonanie szybkiej analizy ryzyka cukrzycy bez konieczności logowania się. \\
\textbf{Warunki wstępne}: Użytkownik niezalogowany, otwarty ekran startowy aplikacji. \\
\textbf{Przebieg}:
\begin{enumerate}
    \item Gość wybiera opcję „Oblicz ryzyko cukrzycy”.
    \item System wyświetla formularz do diagnozy.
    \item Gość uzupełnia formularz i zatwierdza go.
    \item System przetwarza dane i wyświetla diagnozę.
    \item System wyświetla interpretację wyniku.
\end{enumerate}

\subsection{Przypadek użycia: Rejestracja}
\textbf{Aktor}: Gość \\
\textbf{Opis}: Utworzenie nowego konta użytkownika. \\
\textbf{Warunki wstępne}: Użytkownik nie posiada konta lub nie jest zalogowany. \\
\textbf{Przebieg}:
\begin{enumerate}
    \item Gość wybiera opcję rejestracji.
    \item System wyświetla formularz do rejestracji.
    \item Gość podaje wymagane dane i zatwierdza wybór.
    \item System weryfikuje poprawność danych i tworzy konto w bazie.
    \item Następuje automatyczne logowanie użytkownika.
\end{enumerate}

\subsection{Przypadek użycia: Logowanie}
\textbf{Aktor}: Gość \\
\textbf{Opis}: Uwierzytelnienie użytkownika w systemie. \\
\textbf{Warunki wstępne}: Użytkownik posiada konto, ale nie jest zalogowany. \\
\textbf{Przebieg}:
\begin{enumerate}
    \item Gość wybiera opcję Logowania.
    \item System wyświetla formularz logowania.
    \item Gość wprowadza login i hasło.
    \item System weryfikuje zgodność danych z bazą bezpiecznie przechowywanych haseł.
    \item[5a.] Dane są poprawne – użytkownik uzyskuje dostęp do aplikacji jako użytkownik zalogowany.
    \item[5b.] Dane są błędne – system wyświetla komunikat o błędzie.
    \item[6b.] Powrót do kroku 3.
\end{enumerate}

\subsection{Przypadek użycia: Symulacja ryzyka oraz porady}
\textbf{Aktor}: Użytkownik Zalogowany \\
\textbf{Opis}: Wykonanie analizy ryzyka cukrzycy z trwałym zapisem wyniku w historii użytkownika. \\
\textbf{Warunki wstępne}: Użytkownik zalogowany w aplikacji. \\
\textbf{Przebieg}:
\begin{enumerate}
    \item Użytkownik wybiera opcję „Oblicz ryzyko cukrzycy”.
    \item System wyświetla formularz do diagnozy.
    \item Użytkownik uzupełnia formularz i zatwierdza go.
    \item System przetwarza dane i wyświetla diagnozę wraz z poradami.
    \item System zapisuje dane w bazie danych.
\end{enumerate}

\subsection{Przypadek użycia: Przeglądanie historii predykcji}
\textbf{Aktor}: Użytkownik Zalogowany \\
\textbf{Opis}: Przeglądanie zapisanych wyników badań. \\
\textbf{Warunki wstępne}: Użytkownik zalogowany, posiada w bazie co najmniej jeden zapisany wynik. \\
\textbf{Przebieg}:
\begin{enumerate}
    \item Użytkownik wybiera moduł historii.
    \item System pobiera historię badań zdrowotnych z bazy danych.
    \item System wyświetla listę wykonanych predykcji.
    \item Użytkownik może wybrać konkretny wpis, aby zobaczyć jego szczegóły.
\end{enumerate}

\chapter{Modele Uczenia Maszynowego}
\label{cha:modele}

\section{Przegląd Modeli}
W celu zapewnienia wysokiej wiarygodności predykcji, system wykorzystuje zespół trzech różnych modeli klasyfikacyjnych działających równolegle. Pozwala to na porównanie wyników i zwiększenie pewności diagnozy.

Wykorzystane algorytmy to:
\begin{enumerate}
    \item \textbf{Logistic Regression}: Model liniowy, służący jako punkt odniesienia (baseline). Charakteryzuje się wysoką intepretowalnością. W eksperymentach osiągnął dokładność 64.33\% przy najkrótszym czasie treningu (0.27s).
    \item \textbf{Random Forest}: Zespół drzew decyzyjnych. Model ten jest odporny na overfitting. Osiągnął dokładność 79.45\% w czasie treningu wynoszącym 25.41s.
    \item \textbf{Gradient Boosting}: Zaawansowana metoda ensemble. Zapewniała najwyższą ogólną dokładność (84.9\%), jednak odbyło się to kosztem najdłuższego czasu treningu (75.34s).
\end{enumerate}

Modele zostały skonfigurowane do klasyfikacji wieloklasowej (0 - brak cukrzycy, 1 - stan przedcukrzycowy, 2 - cukrzyca) lub binarnej, w zależności od konfiguracji treningowej. W obecnej wersji system wspiera klasyfikację w 3 klasach.

\section{Dane Treningowe i Preprocessing}
Modele zostały wytrenowane na zbiorze danych pochodzącym z CDC BRFSS (Behavioral Risk Factor Surveillance System), zawierającym ok. 250,000 rekordów~\cite{kaggle2021}.

\subsection{Cechy wejściowe}
Każdy model przyjmuje na wejściu wektor 18 cech opisujących stan zdrowia i styl życia pacjenta.

\begin{table}[htbp]
    \centering
    \tiny
    \caption{Cechy wejściowe modelu}
    \label{tab:cechy}
    \begin{tabularx}{\textwidth}{|l|l|X|}
        \hline
        \textbf{Kategoria} & \textbf{Cecha} & \textbf{Opis} \\
        \hline
        Demografia & Sex, Age & Płeć i wiek (kategorie) \\
        \hline
        Badania & BMI, HighBP, HighChol & Wskaźnik masy ciała, nadciśnienie, wysoki cholesterol \\
        \hline
        Zdrowie & GenHlth, PhysHlth, MentHlth & Ogólna ocena zdrowia, dni złego samopoczucia fiz./psych. \\
        \hline
        Choroby & Stroke, HeartDiseaseorAttack & Przebyty udar, choroby serca \\
        \hline
        Nawyki & Smoker, Alcohol, Fruits, Veggies & Palenie, alkohol, dieta \\
        \hline
        Inne & PhysActivity, DiffWalk & Aktywność fizyczna, trudności w chodzeniu \\
        \hline
        Opieka & AnyHealthcare, NoDocbcCost & Dostęp do opieki, koszt wizyt \\
        \hline
    \end{tabularx}
\end{table}

\subsection{Preprocessing}
Przed podaniem danych do modeli zastosowano następujące kroki przetwarzania wstępnego:
\begin{enumerate}
    \item \textbf{Czyszczenie danych}: Usunięcie brakujących wartości i duplikatów.
    \item \textbf{Skalowanie}: Zastosowano \texttt{StandardScaler} do normalizacji cech numerycznych (np. BMI), aby sprowadzić je do wspólnej skali (średnia 0, odchylenie standardowe 1).
    \item \textbf{Balansowanie klas}: Zbiór treningowy był niezbalansowany (znacznie więcej osób zdrowych). Zastosowano parametr \texttt{class\_weight='balanced'} w modelach oraz techniki oversamplingu (podczas eksperymentów) w celu wyrównania szans dla klas mniejszościowych (cukrzyca).
\end{enumerate}

\section{Szczegółowa Ewaluacja i Problem Niezbalansowania}
Modele zostały zweryfikowane na zbiorze testowym liczącym 50 736 próbek. Kluczowym wyzwaniem w tym zbiorze danych jest silne niezbalansowanie klas – zdecydowana większość próbek (ok. 84\%) to osoby zdrowe (klasa 0).

\subsection{Pułapka Dokładności (Accuracy Paradox)}
Analizując wyniki, można zauważyć, że **Gradient Boosting** osiągnął najwyższą dokładność (84.9\%). Jest to jednak wynik mylący. Model ten zoptymalizował się pod klasę większościową, niemal całkowicie ignorując klasy rzadkie, które są najważniejsze z medycznego punktu widzenia.
\begin{itemize}
    \item Recall dla stanu przedcukrzycowego (klasa 1) w Gradient Boosting wyniósł 0.00 (model w ogóle jej nie wykrywa).
    \item Recall dla cukrzycy (klasa 2) wyniósł jedynie 0.18.
\end{itemize}

Dlatego w systemach medycznych ważniejsza od ogólnej dokładności jest zdolność do detekcji choroby (Czułość/Recall).

\textbf{Logistic Regression}, mimo najniższej ogólnej dokładności (64.33\%), wykazał się relatywnie najlepszą zdolnością do wykrywania trudnych przypadków (Recall dla klasy 2 wynosi 0.59, a dla klasy 1 wynosi 0.30). Oznacza to, że częściej generuje fałszywe alarmy (niższa precyzja), ale rzadziej przeoczy osobę chorą.

\textbf{Random Forest} stanowi kompromis, oferując lepszą precyzję niż regresja i znacznie lepszą wykrywalność chorób niż Gradient Boosting.

\begin{table}[htbp]
    \centering
    \caption{Porównanie zdolności detekcji Cukrzycy (Klasa 2) - klasy rzadkiej}
    \label{tab:metrics_diabetes}
    \begin{tabular}{|l|c|c|c|}
        \hline
        \textbf{Model} & \textbf{Precision} & \textbf{Recall (Czułość)} & \textbf{F1-Score} \\
        \hline
        Logistic Regression & 0.35 & \textbf{0.59} & \textbf{0.44} \\
        \hline
        Random Forest & 0.38 & 0.48 & 0.43 \\
        \hline
        Gradient Boosting & \textbf{0.56} & 0.18 & 0.27 \\
        \hline
    \end{tabular}
\end{table}

Podane czasy (0.27s vs 75.34s) odnoszą się do procesu \textbf{trenowania} modelu, a nie predykcji. Czas predykcji dla pojedynczego pacjenta we wszystkich modelach jest liczony w milisekundach i nie stanowi wąskiego gardła. Zastosowanie trzech modeli (Ensemble) w aplikacji pozwala zniwelować słabości pojedynczych algorytmów.

\section{Implementacja Predykcji}
Proces predykcji w systemie przebiega następująco (listing \ref{lst:predykcja}):

\begin{lstlisting}[style=stylePython, caption=Funkcja predykcji w backendzie, label=lst:predykcja]
def predict_diabetes_risk(data):
    # 1. Konwersja danych do DataFrame
    input_df = pd.DataFrame([data])
    
    # 2. Skalowanie danych (uzycie zapisanego scalera)
    input_scaled = scaler.transform(input_df)
    
    # 3. Iteracja przez modele
    results = {}
    for name, model in models.items():
        # Predykcja klasy i prawdopodobienstw
        prediction = model.predict(input_scaled)[0]
        probs = model.predict_proba(input_scaled)[0]
        
        results[name] = {
            'prediction': int(prediction),
            'probabilities': probs.tolist()
        }
    
    return results
\end{lstlisting}

\chapter{Wyjaśnialność i Integracja AI}
\label{cha:ai}

\section{Wyjaśnialność Modeli (XAI) z SHAP}
Modele uczenia maszynowego często działają jak "czarne skrzynki". Aby zwiększyć zaufanie użytkownika do diagnozy, system implementuje metody XAI (Explainable AI), konkretnie bibliotekę SHAP (SHapley Additive exPlanations).

Dla każdej predykcji system oblicza wartości SHAP, które pokazują, jak każda cecha wpłynęła na wynik końcowy (zwiększając lub zmniejszając ryzyko).

\subsection{Implementacja SHAP}
Analiza odbywa się w czasie rzeczywistym po wykonaniu predykcji:

\begin{lstlisting}[style=stylePython, caption=Generowanie wyjasnien SHAP, label=lst:shap]
def get_shap_explanation(model, input_scaled_df):
    # Utworzenie explainera dla modelu drzewiastego
    explainer = shap.TreeExplainer(model)
    shap_values = explainer.shap_values(input_scaled_df)
    
    # Przetwarzanie wynikow...
    # Identifikacja czynnikow ryzyka (wplyw dodatni)
    # Identifikacja czynnikow ochronnych (wplyw ujemny)
    
    return risk_factors, protective_factors
\end{lstlisting}

Wyniki analizy SHAP są przekazywane do modelu językowego w celu wygenerowania bardziej precyzyjnych i spersonalizowanych zaleceń. Dzięki temu porada zdrowotna uwzględnia konkretne czynniki, które zaważyły na wyniku, takie jak BMI czy nadciśnienie.

\section{Generatywna AI - Google Gemini}
Oprócz twardych danych liczbowych, system oferuje "ludzką" poradę generowaną przez duży model językowy (LLM). Projekt wykorzystuje API Google Gemini.

\subsection{Prompt Engineering}
Kluczem do uzyskania wartościowych porad jest odpowiednio skonstruowany prompt. Aplikacja dynamicznie buduje zapytanie do modelu, zawierające:
\begin{enumerate}
    \item Dane pacjenta (wiek, płeć, wyniki badań).
    \item Wynik predykcji (ryzyko cukrzycy).
    \item Kontekst roli ("Jesteś asystentem medycznym...").
    \item Ograniczenia ("Nie stawiaj ostatecznej diagnozy, sugeruj konsultację lekarską").
\end{enumerate}

\begin{lstlisting}[style=stylePython, caption=Integracja z Gemini API, label=lst:gemini]
def generate_llm_advice(data, prediction_result):
    prompt = f"""
    Jako ekspert medyczny, przeanalizuj nastepujacy przypadek:
    Pacjent: Kobieta, 45 lat, BMI 28.
    Wynik modelu ML: Wysokie ryzyko cukrzycy (85%).
    
    Podaj 3 konkretne kroki, ktore pacjent moze podjac, 
    aby zmniejszyc ryzyko. Uzywaj empatycznego jezyka.
    """
    
    response = model.generate_content(prompt)
    return response.text
\end{lstlisting}

Dzięki temu użytkownik otrzymuje nie tylko suchy wynik "Ryzyko: Wysokie", ale także spersonalizowany plan działania (np. "Zwiększ aktywność fizyczną do 30 min dziennie, skonsultuj poziom cukru z lekarzem POZ").
\chapter{Implementacja i API}
\label{cha:implementacja}

\section{Backend (Flask)}
Backend aplikacji pełni rolę huba integrującego bazę danych, modele ML oraz zewnętrzne API. Został zrealizowany w mikro-frameworku Flask, co zapewnia lekkość i łatwość rozbudowy.

\subsection{Struktura API}

Główne endpointy API to:

\begin{itemize}
    \item \texttt{POST /api/predict} - Główny endpoint. Przyjmuje JSON z danymi ankiety, zwraca wyniki ze wszystkich 3 modeli, analizę SHAP oraz poradę AI.
    \item \texttt{POST /api/auth/register} - Rejestracja nowego użytkownika.
    \item \texttt{POST /api/auth/login} - Logowanie (zwraca token JWT access + refresh).
    \item \texttt{GET /api/user/history} - Pobranie historii predykcji zalogowanego użytkownika.
\end{itemize}

\subsection{Model Danych (SQLAlchemy)}
Baza danych przechowuje informacje o użytkownikach oraz historię ich badań. Wykorzystano ORM SQLAlchemy.

Relacje:
\begin{itemize}
    \item \texttt{User} (1) $\longleftrightarrow$ (N) \texttt{PredictionResult}
    \item \texttt{User} (1) $\longleftrightarrow$ (1) \texttt{UserProfile} (dane demograficzne)
\end{itemize}

\section{Frontend (React)}
Warstwa prezentacji została zbudowana jako Single Page Application. Wykorzystuje:
\begin{itemize}
    \item \textbf{React Router}: Do obsługi nawigacji bez przeładowywania strony.
    \item \textbf{Axios}: Do komunikacji z API.
    \item \textbf{Context API}: Do zarządzania stanem sesji użytkownika (autentykacja).
\end{itemize}

Proces badania składa się z 3 kroków (formularz wieloetapowy):
1. Podstawowe dane (Wiek, Płeć).
2. Parametry zdrowotne (BMI, Nadciśnienie).
3. Styl życia (Dieta, Używki).

Poniżej przedstawiono fragment komponentu formularza obsługującego wysyłkę danych:

\begin{lstlisting}[style=javaStyle, caption=Obsluga formularza w React, label=lst:react]
const handleSubmit = async (formData) => {
    try {
        const response = await api.post('/predict', formData);
        setResult(response.data);
        navigate('/results');
    } catch (error) {
        console.error("Blad predykcji:", error);
    }
};
\end{lstlisting}

\subsection{Widoki Aplikacji}
Interfejs użytkownika został zaprojektowany z myślą o przejrzystości i łatwości obsługi. Poniżej przedstawiono kluczowe ekrany aplikacji.

\subsubsection{Strona Główna i Formularz}
Strona główna wita użytkownika i wyjaśnia cel aplikacji. Głównym elementem jest wieloetapowy formularz (Wizard), który prowadzi pacjenta przez proces wprowadzania danych.

\begin{figure}[H]
    \centering
    \includegraphics[width=1.0\textwidth]{figures/landingpage.png}
    \caption{Interfejs strony głównej}
    \label{fig:home}
\end{figure}

Kroki formularza zostały zaprojektowane tak, aby były intuicyjne i nie przytłaczały ilością pytań na jednym ekranie.

\begin{figure}[H]
    \centering
    \includegraphics[width=1.0\textwidth]{figures/formularz.png}
    \caption{Widok kroków formularza diagnostycznego}
    \label{fig:form_steps}
\end{figure}

\subsubsection{Prezentacja Wyników}
Po przetworzeniu danych, użytkownik otrzymuje czytelny wynik w formie graficznej (wskaźnik ryzyka) oraz tekstowej. Kolorystyka (zielony/żółty/czerwony) natychmiastowo informuje o poziomie zagrożenia.

\begin{figure}[H]
    \centering
    \includegraphics[width=1.0\textwidth]{figures/wynikguest.png}
    \caption{Ekran wyników predykcji z wykresem ryzyka}
    \label{fig:results}
\end{figure}

\subsubsection{Analiza Szczegółowa (SHAP i AI)}
Dla bardziej dociekliwych użytkowników dostępna jest sekcja szczegółowa, zawierająca spersonalizowaną poradę wygenerowaną przez asystenta AI.

\begin{figure}[H]
    \centering
    \includegraphics[width=1.0\textwidth]{figures/wynikiuserloggewdin.png}
    \caption{Szczegółowa analiza czynników ryzyka i porada AI}
    \label{fig:shap_ai}
\end{figure}

\subsubsection{Panel Użytkownika i Historia}
Zalogowani użytkownicy mają dostęp do panelu, w którym mogą zarządzać swoim profilem oraz przeglądać historię wykonanych badań.

\begin{figure}[H]
    \centering
    \includegraphics[width=1.0\textwidth]{figures/userpage.png}
    \caption{Panel użytkownika}
    \label{fig:user_panel}
\end{figure}

Sekcja historii pozwala na śledzenie zmian ryzyka w czasie, co jest kluczowe dla prewencji długoterminowej.

\begin{figure}[H]
    \centering
    \includegraphics[width=1.0\textwidth]{figures/historiapredykcji.png}
    \caption{Widok historii predykcji}
    \label{fig:history}
\end{figure}

\section{Podsumowanie}
Stworzony system \textit{DiabetesRiskPredictor} stanowi kompleksowe rozwiązanie demonstrujące praktyczne zastosowanie uczenia maszynowego w medycynie prewencyjnej. Połączenie klasycznych modeli klasyfikacyjnych z nowoczesną generatywną AI pozwala na dostarczenie użytkownikowi wartościowej i zrozumiałej informacji zwrotnej.


% Wyłączenie działania `ulem` na czas bibliografii
\renewcommand{\emph}[1]{\textit{#1}}
% Bibliografia
% Dodanie bibliografi do spisu treści
\addcontentsline{toc}{section}{\textbf{Bibliografia}}
\bibliographystyle{unsrturl}
\bibliography{bibliografia}

% Przywrócenie działania `ulem`
\renewcommand{\emph}[1]{\uline{#1}}

\clearpage
% Dodanie spisu rysunków do spisu treści
\addcontentsline{toc}{section}{\textbf{Spis rysunków}}
\listoffigures
\clearpage

% Dodanie spisu tabel do spisu treści
\addcontentsline{toc}{section}{\textbf{Spis tabel}}
\listoftables
\clearpage

% Dodanie spisu listingow do spisu treści
\addcontentsline{toc}{section}{\textbf{Spis listingów}}
\lstlistoflistings
\clearpage






% \appendix
\chapter*{Streszczenie}
\label{cha:streszczenie}
\makeatletter
\addcontentsline{toc}{section}{\textbf{Streszczenie}}

Niniejszy projekt, \textit{DiabetesRiskPredictor}, to nowoczesna aplikacja webowa służąca do przewidywania ryzyka wystąpienia cukrzycy typu 2. System integruje zaawansowane modele uczenia maszynowego (Logistic Regression, Random Forest, Gradient Boosting) z interaktywnym interfejsem użytkownika. Aplikacja nie tylko klasyfikuje ryzyko, ale również dostarcza spersonalizowanych wyjaśnień wyników dzięki wykorzystaniu metodologii SHAP (SHapley Additive exPlanations) oraz generatywnej sztucznej inteligencji (Google Gemini API).

Część serwerowa (Backend) została zaimplementowana w języku Python z użyciem frameworka Flask, natomiast część kliencka (Frontend) wykorzystuje bibliotekę React. Projekt kładzie duży nacisk na wyjaśnialność (XAI) oraz edukację użytkownika w zakresie profilaktyki zdrowotnej.
\include{appendix/statement-A}


\end{document}
